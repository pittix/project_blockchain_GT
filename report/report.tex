\documentclass[conference]{IEEEtran}
\IEEEoverridecommandlockouts
% The preceding line is only needed to identify funding in the first footnote. If that is unneeded, please comment it out.
\usepackage{cite}
\usepackage{amsmath,amssymb,amsfonts}
\usepackage{algorithmic}
\usepackage{graphicx}
\usepackage{textcomp}
\usepackage{xcolor}
\def\BibTeX{{\rm B\kern-.05em{\sc i\kern-.025em b}\kern-.08em
    T\kern-.1667em\lower.7ex\hbox{E}\kern-.125emX}}
\begin{document}

\title{A game-theoretical analysis of BATMAN protocol}

\author{\IEEEauthorblockN{Andrea Pittaro}
\IEEEauthorblockA{\textit{Dipartimento di Ingegneria dell'Informazione} \\
Padova, Italy \\
andrea.pittaro@studenti.unipd.it}
\and
\IEEEauthorblockN{Enrico Lovisotto}
\IEEEauthorblockA{\textit{Dipartimento di Ingegneria dell'Informazione} \\
Padova, Italy \\
enrico.lovisotto@studenti.unipd.it}
}

\maketitle

\begin{abstract}

\end{abstract}

\section{Introduction}

\section{State of the art}

BATMAN overview
- history
- how does the protocol work: principles

\section{Methodology}

Analysis via simulation
- event-driven simulator
- hypothesis, simplifications introduced
- diagrams and stuff with descriptions and assumptions

Expected results given the setup
(ex. worst network performance for unfair users)

\section{Results}

Network performances between fair and unfair users varying
- percentage of unfair users
- requested traffic (number and intensity of apps)
- channel quality

\section{Conclusion}

\bibliography{report}
\bibliographystyle{IEEEtran}

\end{document}

%%% Local Variables:
%%% mode: latex
%%% TeX-master: t
%%% End:
