\documentclass[conference]{IEEEtran}
\IEEEoverridecommandlockouts
% The preceding line is only needed to identify funding in the first footnote. If that is unneeded, please comment it out.
\usepackage{cite}
\usepackage{amsmath,amssymb,amsfonts}
\usepackage{algorithmic}
\usepackage{graphicx}
\usepackage{hyperref}
\usepackage{glossaries}
\usepackage{textcomp}
\usepackage{xcolor}
\def\BibTeX{{\rm B\kern-.05em{\sc i\kern-.025em b}\kern-.08em
    T\kern-.1667em\lower.7ex\hbox{E}\kern-.125emX}}

\newacronym{batman}{B.A.T.M.A.N.}{Better Approach To Mobile Adhoc Networking}

\newacronym{rtt}{RTT}{Round Trip Time}
\newacronym{arq}{ARQ}{Automatic Repeat-reQuest}
\newacronym{ne}{NE}{Nash Equilibrium}


\begin{document}

\title{A game-theoretical analysis of BATMAN-adv protocol}

\author{\IEEEauthorblockN{Andrea Pittaro}
\IEEEauthorblockA{\textit{Dipartimento di Ingegneria dell'Informazione} \\
Padova, Italy \\
andrea.pittaro@studenti.unipd.it}
\and
\IEEEauthorblockN{Enrico Lovisotto}
\IEEEauthorblockA{\textit{Dipartimento di Ingegneria dell'Informazione} \\
Padova, Italy \\
enrico.lovisotto@studenti.unipd.it}
}

\maketitle

\begin{abstract}

\end{abstract}

\section{Introduction}

\section{State of the art}

\textbf{BATMAN overview}\\

- history\\
\gls{batman} was born in 2007 as a mesh networks protocol that allowed users to create network using their devices as nodes that routes traffic and transmit data to the network. In 2008 \gls{batman} became \gls{batman}-adv breaking the ISO/OSI stack and exploiting the knowledge at layer 2 and 3 to efficiently find paths and route the packets arriving to each node. During these years the protocol evolved and in 2011 became officially supported by the Linux Kernel, allowing all PCs to become a node in the network. Since then, the protocol was widely tested, thanks to the Freifunk initiative, which aims to bring to everyone a decentralized WiFi network and it's spread to all Germany.

- how does the protocol work: principles
The protocol \gls{batman}-adv creates a network of Ad-Hoc connections where each node informs the neighbours about his presence using an announcement message, and, after it joins the network, it will need two tables: a local one and a global one. The local table is used to route each packet, generated or arriving from a node, to the best next hop, according to a metric. Each node will repeat the process of routing the packet to the best hop until it arrives to the destination node. The global table, instead, is used to store the table of all or part of the network, so that if a node fails, the other one can exploit it to find new paths. This allows the network to be resilient and more reliable to a node that can move geographically, leave the network or join. In the latter case, the node can initially fill the table with the neighbour to contact when it wants to send some packets. 


\section{Methodology}

\subsection{Network setup}

In order to assess and verify the expected properties of the \gls{batman}
protocol, we decided to run an event-driven simulation.

So, a certain number of entities were scattered uniformly across the simulation
area, as shown in \autoref{fig:nodes}. This strategy allows to change topology
simply switching the generator random seed and will be used for validating all our results.

\begin{figure}[h]
  \centering
  % \includegraphics[]{TODO}
  \caption{Nodes, labeled by their IP, are connected each other wirelessly.}
  \label{fig:nodes}
\end{figure}

Once placed, each couple of nodes close enough were linked by means of a
wireless fully reliable channel, characterized by a \gls{rtt} and a
retransmission probability $p_r$, both function of the reciprocal distance.

This assumption on the channel characteristics reduces the model complexity
while keeping an high degree of realism, as \gls{arq} strategies are common
practice at all protocol stack levels.

\begin{equation}
  \begin{split}
    p_r & = e^{-\frac{d}{D}} \\
    RTT &= 2 \frac{d}{c_0} + t_{proc}
  \end{split}
\end{equation}

Given previous hypothesis, communication can be described by message exchange in
a directed graph made of \gls{batman}, channels and application layers, as shown
in \autoref{fig:graph}.

\begin{figure}[h]
  \centering
  % \includegraphics[]{TODO}
  \caption{Abstract graph spanning all logical components of the network.}
  \label{fig:graph}
\end{figure}

\subsection{Expected results}

In order to be as general as possible, our network connects memoryless sources
of traffic communicating to each other: the \gls{batman} layer has the goal to
serve its users as best as it can. To do so, node have to rely to a certain
extent on their neighbours, since not all destinations are reachable in a single
hop.

From a game theoretical point of view, two different strategies can be chosen by
each layer in order to maximize its objective: either it can collaborate with
its own kind, using part of its bandwidth to forward packets of other sources,
or selfishly transmit only its own packets.

As we will show in the upcoming \autoref{sec:results}, the protocol is designed
in such a way that the altruistic path is more convenient for the nodes to
pursue. The collaboration will turn out to be, in fact, a \gls{ne} for the game
played between all the connected entities.

We will furthermore analyze in which terms the network can sustain itself when a
growing fraction of its users are selfish or network conditions degrade.

% TODO bit of spoilers for results section

\section{Results} \label{sec:results}

Network performances between fair and unfair users varying
- percentage of unfair users
- requested traffic (number and intensity of apps)
- channel quality

\section{Conclusion}

\bibliography{report}
% \bibliographystyle{IEEEtran}

\end{document}

%%% Local Variables:
%%% mode: latex
%%% TeX-master: t
%%% End:
